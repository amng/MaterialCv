%%!TEX program = xelatex
%%========================================================================================================
%% Material CV
%% Author: Artur Gomes
%%
%% Description: This template design was based on the
%% Google Material Design for the new Android 5.0 L 
%% version.
%%
%% Aknowledgements
%% The font used was the font Roboto used in Android:
%% http://developer.android.com/design/style/typography.html
%% 
%% Used the elegant font iconset for icons (colors were changed acording to the color of the separator):
%% http://www.flaticon.com/packs/elegant-font
%%========================================================================================================
% http://tex.stackexchange.com/questions/140960/custom-shadow-border-effect-with-tikz

\documentclass{article}
\usepackage{tikz}
%\usepackage{fontspec}
\usepackage[sfdefault]{roboto}
\usepackage[top=0cm, bottom=2cm, outer=0cm, inner=0cm]{geometry}
\usepackage[pages=some]{background}
\usepackage{materialCard}
\usepackage{multicol}

\backgroundsetup{scale=1, color=black, opacity=1, placement=top, angle=0, contents={%
  \includegraphics[width=\paperwidth,height=150pt]{images/pic}
  }}
\begin{document}
\begin{minipage}[t]{\paperwidth}
	\pagenumbering{gobble}
	\vspace*{50pt} %spacing used to move the first and last name down
	\Large
	\BgThispage
\end{minipage}
%\LARGE
%\noindent\colorbox{MaterialGreen}
%{\parbox[c][25pt][c]{\textwidth}{\hspace{15pt}\textcolor{white}{Contacts}}} %%Contacts separator
\hspace*{10pt} 
\begin{minipage}[t]{\paperwidth}
	\vspace{75pt}
	\large
	\vspace{5pt}
	\begin{multicols}{2}
	\materialCard[topimage=images/profile,topcolor=MaterialRed500, cardwidth=0.2\paperwidth, topheight=68pt]{%!TEX root = material_cv.tex

\large
\hspace*{70pt}
\begin{minipage}[t]{0.2\paperwidth}
\includegraphics[width=4.5cm]{images/profile}

\vspace{10pt}
\Large
\hspace*{33pt}\textcolor{white}{1st last Name}\\
\hspace*{33pt}\small \textcolor{MaterialGrey200}{Position}
\end{minipage}}
	\materialCard[title=Contacts, topcolor=MaterialGreen500, cardwidth=0.67\paperwidth]{%!TEX root = cv.tex

\newcommand{\ContactEntry}[2]{
	\normalsize
	$\begin{array}{l}
	{\includegraphics[height=10pt]{#1}}
	\end{array}
	$ #2
}


%\LARGE
%\noindent\colorbox{materialGreen}
%{\parbox[c][25pt][c]{\textwidth}{\hspace{15pt}\textcolor{white}{Contacts}}} %%Contacts separator
\begin{minipage}[t]{0.66\textwidth}
	\vspace{1pt}
	\begin{multicols}{2}
		%%Contact Information
		%Add new or alter contact entries in this section by using the examples below
		%for other contact information search the image folder for other icons
		\ContactEntry{images/green/ic_phone}{111-222-3333}

		\ContactEntry{images/green/ic_email}{mail@mail.com}

		\ContactEntry{images/green/ic_web}{http://mywebsite.com}

		\ContactEntry{images/green/ic_people}{https://www.linkedin.com/in/yourname}

		\columnbreak

		\ContactEntry{images/green/ic_home}{Street, number

		\hspace{22pt}City, State 0000-0000

		\hspace{22pt}Country}
	\end{multicols}
	\vspace{0.1pt}
\end{minipage}}
	\end{multicols}{}

	\large
	\materialCard[title=Experience, topcolor=MaterialRed500]{%!TEX root = cv.tex

%%Command used in order to make each entry cleaner
\normalsize
\newcommand{\ExperienceEntry}[5]{
\begin{tabular}{  p{\dimexpr 0.15\linewidth-2\tabcolsep} 
                   p{\dimexpr 0.75\linewidth-2\tabcolsep}}
  \textbf{\textcolor{MaterialRed900}{#1-}}\textbf{\textcolor{MaterialRed700}{#2}} & \textbf{#3}\\
  	& \normalsize #4\\
  	&{\small \textcolor{MaterialGrey500}{#5}}
\end{tabular}
}

%\LARGE
%\noindent\colorbox{materialRed}
%{\parbox[c][25pt][c]{\textwidth}{\hspace{15pt}\textcolor{white}{Experience}}} %%Contacts separator
\begin{minipage}[t]{0.85\linewidth}
%%Contact Information
\normalsize
%Add new or alter education entries in this section by using the examples below
%\ExperienceEntry{starting year}{final year}{Position}{position description if applicable}{Place}
\ExperienceEntry{2000}{2014}{Position 0}{Job description....}{Place x}
\ExperienceEntry{2000}{2014}{Position 0}{Job description....}{Place x}
\end{minipage}
}{}

	\large
	\vspace{5pt}
	\materialCard[title=Education, topcolor=MaterialBlue500]{%!TEX root = material_cv.tex

%%Command used in order to make each entry cleaner
\normalsize
\newcommand{\EducationEntry}[5]{
\begin{tabular}{ p{\dimexpr 0.15\linewidth-2\tabcolsep} 
                   p{\dimexpr 0.75\linewidth-2\tabcolsep}}
  \textbf{\textcolor{MaterialBlue900}{#1-}}\textbf{\textcolor{MaterialBlue500}{#2}} & \textbf{#3}\\
  	& \normalsize #4\\
  	&{\small \textcolor{MaterialGrey500}{#5}}
  	\vspace*{2pt}
\end{tabular}
}

%\LARGE
%\noindent\colorbox{MaterialBlue500}
%{\parbox[c][25pt][c]{\textwidth}{\hspace{15pt}\textcolor{white}{Education}}} %%Contacts separator
\begin{minipage}[t]{0.85\linewidth}
%%Contact Information
\normalsize
%Add new or alter education entries in this section by using the examples below
%\EducationEntry{starting year}{final year}{Type of studies}{Studies description if applicable}{Place of studies}
\EducationEntry{2000}{2014}{Studies in something 1}{Studies description....}{Place X}
\EducationEntry{2014}{2017}{Studies in something 1}{Studies description....}{Place X}
\end{minipage}}


	\large
	\vspace{5pt}
	\materialCard[title=Interests, topcolor=MaterialAmber500]{%!TEX root = cv.tex
\newcommand{\textBox}[1]{
\begin{tabular}{ p{\dimexpr 0.9\linewidth-4\tabcolsep} }
  	{\normalsize #1}
\end{tabular}
\vspace{1pt}
}

%\LARGE
%\noindent\colorbox{MaterialAmber500}
%{\parbox[c][25pt][c]{\textwidth}{\hspace{15pt}\textcolor{white}{Interests}}} %%Contacts separator

%%Contact Information
\large
\begin{minipage}[t]{0.83\linewidth}
\textcolor{MaterialAmber500}{\textbf{Profes}}\textcolor{MaterialGrey700}{\textbf{sional:}}\\
\textBox{
professional interests.... %%edit professional interest here
}\\


\textcolor{MaterialAmber500}{\textbf{Pers}}\textcolor{MaterialGrey700}{\textbf{onal:}}\\
\textBox{
personal interests....%%edit professional interest here
}\\
\end{minipage}}

	\large
	\vspace{5pt}
	\materialCard[title=Awards, topcolor=MaterialPurple500]{%!TEX root = material_cv.tex

%%Command used in order to make each entry cleaner
\normalsize
\newcommand{\AwardEntry}[4]{
\begin{tabular}{p{\dimexpr 0.15\linewidth-2\tabcolsep} 
                p{\dimexpr 0.50\linewidth-2\tabcolsep}
                p{\dimexpr 0.25\linewidth-2\tabcolsep}}
  \textbf{\textcolor{MaterialPurple500}{#1}}\hspace{20pt} & \textbf{#2} & {\small \textcolor{MaterialGrey500}{#4}}\\
  	& \normalsize #3
  \vspace*{2pt}
\end{tabular}
}

%\LARGE
%\noindent\colorbox{MaterialPurple500}
%{\parbox[c][25pt][c]{\textwidth}{\hspace{15pt}\textcolor{white}{Awards}}} %%Contacts separator

%%Contact Information
\begin{minipage}[t]{0.85\linewidth}
%Add new or alter education entries in this section by using the examples below
%\EducationEntry{starting year}{final year}{Type of studies}{Studies description if applicable}{Place of studies}
\AwardEntry{2000}{Studies in something 1}{Studies description....}{Place X}
\AwardEntry{2000}{Studies in something 1}{Studies description....}{Place X}
\end{minipage}}
\end{minipage}
%\materialCard[title=Mater222222222222222222222222222ial123Card, image=MaterialRed500]{\begin{tabular}{c} This node \\ is \\ valuable \end{tabular}}{card2}\\\\
\end{document}